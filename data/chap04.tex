%!TEX root = ../main.tex
\chapter{2.5维谐振腔的制备与测量} % (fold)
\label{cha:2_5维谐振腔的制备与测量}
	


        \section{器件测量与制备工艺概述} % (fold)
        \label{sec:制备工艺概述}

            由于2.5维谐振腔的电容部分由上下两层金属以及中间的介电层组成,工艺较为复杂,需要至少三步光刻来完成。而细小的电感部分则需要电子束曝光(EBL)来定义形状,再由蒸发镀膜完成。因此,总的制备工艺大致如下:

            \begin{enumerate}
                \item 通过光刻,磁控溅射蒸镀金属Nb,Lift off三步制作平面波导与三层电容的第一层
                \item 通过ALD生长Al$_2$O$_3$,或通过PECVD生长SiO$_2$或SiN$_x$作为电容三层结构中的介电层
                \item 光刻定义掩膜后通过ICP刻蚀或湿法刻蚀介电层
                    \begin{enumerate}
                        \item 显影后剩余光刻胶应完全盖住三层电容的第一层金属以及上方的介电层
                        \item 刻蚀后剩余的介电层仅遮盖住三层电容的底层,其余电介质需被刻蚀以点焊与继续制作其他结构
                        \item 去除介电层上方残存的光刻胶
                    \end{enumerate}
                \item 利用光刻制作出三层电容的顶层图案并蒸镀金属,对准精度约1微米,Lift off
                \item 利用EBL制作微小电感图案
                    \begin{enumerate}
                        \item 电感图案将与三层电容的上下级板相连
                        \item 对准精度 100-1000 nm
                    \end{enumerate}
                \item 旋涂光刻胶,将器件切为4mm~x~7mm大小,清洁器件
                \item 通过点焊将器件与PCB板线路相连接
            \end{enumerate}


            每一步制备工艺的具体步骤与相关参数在附录\ref{cha:fabrication}中给出。
            
        % subsection 制备工艺概述 (end)

        \section{光刻板的设计与改进} % (fold)
        \label{sec:光刻板的设计}

            由于制备工艺需要三步光刻,对于完整制备一个器件来说,需要的光刻板的图案为一组三个,分别对应\ref{sec:制备工艺概述}制备工艺概述中的前三步。第一步完成二维平面波导以及接地平面的制作,以及三维电容的第一层金属。第二步在生长介电材料后通过光刻制作掩膜覆盖住需要保留的电容的第二层介电层部分,并刻蚀掉没有被覆盖掉的电介质。第三部制作出电容最上层的金属。最后一步通过电子束曝光与蒸镀制作微小的电感部分,并各自与三维电容的上、下两层连接起来。光刻的三个步骤所需的模板作图如图\ref{fig:one_group}所示
            

            \begin{figure}[h]
                \centering
                \includegraphics[width=3in]{one_group.png}
                \caption{制备一个器件所需的三个光刻步骤对应的一组三个模板图案}
                \label{fig:one_group}
            \end{figure}

            \begin{figure}[h]
                \centering
                \includegraphics[width=5in]{LC.png}
                \caption{完整的光刻板图案}
                \label{fig:LC}
            \end{figure}

            由于没有这类谐振腔的制备经验,对于其频率的估算并没有太多把握,因此我们希望有尽可能多的谐振频率与设计相对应的数据,来辅助下一步对仿真结果的修正与器件制备的改进。由于对电容和电感估值的准确度均有待确认,我们需要至少两组不同的器件设计的组合,一组固定$L$变化$C$的大小,另一组固定$C$变化$L$的大小,这样可以通过拟合确定出每个器件的$L$与$C$的值。因此,一个二维平面波导可以与多个谐振腔耦合,可方便测量。另一方面,考虑到电子束曝光难度与耗时均较大,我们打算先尝试中等数值的电容与电感组合,使电感的尺寸能够通过光刻制备,这样即可在三层电容制备的最后一步同时制作出电感,省去了电子束曝光制备电感的步骤,大大加快样品制备与测试速度。解决三层电容的制备后,再通过电子束曝光制作电感。

            综合上述讨论,我们总共设计了若干种不同的模板几何形状,如图\ref{fig:LC}所示,覆盖了较大范围的电容与电感值,为第一次摸索器件制备工艺以及尝试性测量提供较大的频率变化区间,尽可能保证能够测到谐振腔的共振频率。


            \section{器件制备情况与改进} % (fold)
            \label{sec:器件制备情况与改进}
                按照上述计划,我进行了器件制备的工作,目前已完成了两轮完整的器件制备流程。

                在第一轮器件制备过程中,发现了许多问题,并依次进行了分析与解决。第一轮器件制备的器件图像如图\ref{fig:FabRound1}所示。

            \begin{figure}[h]
                \centering
                \includegraphics[width=\textwidth]{FabRound1.png}
                \caption{第一轮器件制备过程}
                \label{fig:FabRound1}
            \end{figure}

            由于PPMS测量系统仅可放下大小为4mm~x~7mm的器件,我们最初即采用该大小的硅片作为衬底开始进行微纳加工。由于器件尺寸较小,在旋涂光刻胶的过程中导致胶厚分布不均,因此光刻第一层图案后效果明显不好,如图\ref{fig:FabRound1}(a)所示。图\ref{fig:FabRound1}(a)中绿色部分为显影后剩余的光刻胶,也即不会被蒸镀上金属的部分。靠上的两个长条形的绿色部分即为平面波导的中心线与两侧接地部分间的间隙,而下面大块绿色部分包围着的即应该为三维电容结构的三层中的第一层。由于光刻胶厚度不均匀,曝光过程中发生了衍射,导致图案变形。因此,我们改用8mm~x~8mm左右大小的硅片进行微纳加工,再在最后点焊前将器件切割为PPMS所要求的大小。进行这步改进后,光刻效果很好,器件的第一层金属顺利完成,对应第\ref{sec:制备工艺概述}节中的步骤1。

            器件制备的第二步,也即生长介电层的步骤较为简单,利用超净间内现有的PECVD或ALD的相关程序即可自动完成。第一轮器件制备选用的介电层为PECVD生长的30nm的SiO$_2$。在进行第三步,也即第二层光刻的显影阶段时,我观察到了器件上有淡蓝色的杂物,疑似为未清洗干净的被曝光的光刻胶,如图\ref{fig:FabRound1}(b)所示,其中浅绿色的小方块即为理应存在剩余光刻胶的部分,成功观察到了完整盖住三维电容第一层金属以及上方介电层的光刻胶。分析后我们认为,残存的光刻胶可能导致该处剩余未被刻蚀的介电层,对器件不会产生本质影响,因此我们暂时忽略这一步的不理想,继续器件制备。进行ICP刻蚀后,器件如图\ref{fig:FabRound1}(c)所示,可以看见电容处的电介质与残留的光刻胶整体呈现深紫色,而图\ref{fig:FabRound1}(b)中的淡蓝色部分仍旧为灰色阴影,应为未刻蚀干净的介电层。图\ref{fig:FabRound1}(c)中还能看到其他两块紫色区域,为第二层光刻对准第一层结构所用的标记符号,因为也没被曝光所以留下了介电层与光刻胶,从而呈紫色。在最后去除介电层上方残留光刻胶时,使用了常用的S1805光刻胶去胶液NMP,但去胶后观察到了剩余的介电层表面有不明灰色纹路,如图\ref{fig:FabRound1}(d)所示。图\ref{fig:FabRound1}(d)即图图\ref{fig:FabRound1}(c)中红色矩形部分去胶后的放大图。这些灰色纹路疑似为未被去净的光刻胶,因此在第二轮器件制备过程中,我们改用PG remover去胶,得到了更好的效果。

            第二、三步完成后,进行最后一步的制作电容的顶层金属。首先进行顶层图案的光刻,光刻的效果如图\ref{fig:FabRound1}(e)与(f)所示,其(f)为(e)中黄色矩形的放大图。可以看出,顶层与底层间有一定的对准误差。需要注意,在将电感部分与顶层金属和底层金属连接上时,需先通过Argon milling去除底层金属表面的氧化层,随后磁控溅射Nb,才能达到良好的电接触。蒸镀完成后,即可去胶使设计的金属结构存留。此时因为与上下两层电容金属极板相连的电感在平面内围成闭合区域,去胶液难以进入。正常情况下可以尝试超声,但目前器件上有较薄的介电层,容易在超声过程中脱落,因此我们小心尝试了超声的功率与持续时间,最终在保持介电层完整的条件下达到了较好的去胶效果,如图\ref{fig:FabRound1}(g)所示。可以看到,绝大部分应被去掉的金属均成功被去掉,除了第二个谐振腔中电感部分的左下角内仍有一小块金属。

            至此,器件制备基本结束,制备出的2.5维谐振腔如图\ref{fig:FabRound1}(h)所示,电容的下层与上层极板间的对准误差大概在1um,与预期相符。灰色方形部分为上下极板间的介电层,细条形为电感。最后旋涂光刻胶,切片并清洗后,即可进行点焊,并放入PPMS进行测量。然而对于第一批制备出的器件并没有测到明显的谐振腔模式。我们分析后认为原因可能有以下两方面:一方面,由于介电材料相对于空气而言有损耗,谐振腔的$Q_i$比常见传输线谐振腔更小,而由于光刻板的设计使谐振腔与传输线中的场间有一段距离的接地平面,有一定屏蔽作用,导致谐振腔与传输线中的模式耦合较弱,$Q_c$较大,综合这两点,预计谐振腔响应的峰形较浅较宽,可能被背景噪声覆盖;另一方面,器件制备过程中最上层与最下层的对齐误差导致电容偏小,谐振腔频率偏高,可能使谐振频率超出PPMS与VNA测量系统的正常测量范围$1\sim 10$GHz。因此,我们下一步改良了光刻板的设计,增大谐振腔与传输线的耦合强度从而增强谐振腔响应的信号。同时开始第二批器件的制备,采用谐振频率变化范围更广的器件组以使得有谐振腔的频率落在可测范围内的概率更大。

            第二批器件的制备过程如图所示。






            % section 器件制备 (end)
            



            \section{谐振腔响应信号的拟合} % (fold)
            \label{sec:谐振腔响应信号的拟合}
            	









