%!TEX root = ../main.tex

\chapter{结论与改进} % (fold)
\label{cha:结论与改进}
	本文首先对基于超导量子比特与固态自旋的混合量子系统的重要性与研究现状。对超导比特与常见的固态自旋的系统进行了简介。

	对于固态系统与谐振腔的耦合,本文基于新型三维谐振腔,二维平面波导谐振腔,以及新型二维谐振腔(2.5维谐振腔)进行了仿真,并得到了与现有文献与相关工作相符的结论。对于2.5维谐振腔,本文进一步对其设计进行了优化,整理了电感的估算方法。

	实验部分,本文对2.5维谐振腔的测量系统编写了控制程序,并在第\ref{cha:PPMS测量系统}章中对测量系统进行了详细介绍。对于固定测试样品所用的样品托进行了改良,并设计了竖直放置样品的样品托与相应的PCB板,为后续实验做准备。

	对2.5维谐振腔样品的制备,本文整理了制备流程以及每一个环节的具体参数,并实际进行了两轮器件制备。在第一轮器件制备中观察到的若干问题在第二轮制备过程中得到了解决。随后对制备的2.5维谐振腔样品进行了测量与讨论。对于尚未测到理想结果提出了三点可能的因素,较为确信地筛选出了其中的主要原因为器件与传输线耦合强度过弱,并以其为依据对器件的设计进行了改良,将在后续器件制备与测量过程中进行验证。

	本文在附录中\ref{cha:SCQubitPrinciple}给出了超导量子比特的基本性质推导,发展历史简介与技术上的若干改良。在附录\ref{cha:fabrication}中给出了具体的微纳加工工艺。在附录\ref{cha:measurement_code}中给出了测量系统的控制代码。


% chapter 结论与改进 (end)