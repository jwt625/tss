%!TEX root = ../main.tex
\thusetup{
  %******************************
  % 注意:
  %   1. 配置里面不要出现空行
  %   2. 不需要的配置信息可以删除
  %******************************
  %
  %=====
  % 秘级
  %=====
  secretlevel={秘密},
  secretyear={10},
  %
  %=========
  % 中文信息
  %=========
  % ctitle={清华大学学位论文 \LaTeX\ 模板\\使用示例文档 v\version},
  ctitle={基于超导量子比特与固态自旋的混合量子系统},
  cdegree={理学学士},
  cdepartment={物理系},
  cmajor={物理学},
  cauthor={蒋文韬},
  csupervisor={宋祎璞副研究员},
  % cassosupervisor={陈文光教授}, % 副指导老师
  % ccosupervisor={某某某教授}, % 联合指导老师
  % 日期自动使用当前时间,若需指定按如下方式修改:
  % cdate={超新星纪元},
  %
  % 博士后专有部分
  cfirstdiscipline={计算机科学与技术},
  cseconddiscipline={系统结构},
  postdoctordate={2009年7月——2011年7月},
  id={编号}, % 可以留空: id={},
  udc={UDC}, % 可以留空
  catalognumber={分类号}, % 可以留空
  %
  %=========
  % 英文信息
  %=========
  etitle={An Introduction to \LaTeX{} Thesis Template of Tsinghua University v\version},
  % 这块比较复杂,需要分情况讨论:
  % 1. 学术型硕士
  %    edegree:必须为Master of Arts或Master of Science(注意大小写)
  %             “哲学、文学、历史学、法学、教育学、艺术学门类,公共管理学科
  %              填写Master of Arts,其它填写Master of Science”
  %    emajor:“获得一级学科授权的学科填写一级学科名称,其它填写二级学科名称”
  % 2. 专业型硕士
  %    edegree:“填写专业学位英文名称全称”
  %    emajor:“工程硕士填写工程领域,其它专业学位不填写此项”
  % 3. 学术型博士
  %    edegree:Doctor of Philosophy(注意大小写)
  %    emajor:“获得一级学科授权的学科填写一级学科名称,其它填写二级学科名称”
  % 4. 专业型博士
  %    edegree:“填写专业学位英文名称全称”
  %    emajor:不填写此项
  edegree={Doctor of Engineering},
  emajor={Computer Science and Technology},
  eauthor={Xue Ruini},
  esupervisor={Professor Zheng Weimin},
  eassosupervisor={Chen Wenguang},
  % 日期自动生成,若需指定按如下方式修改:
  % edate={December, 2005}
  %
  % 关键词用“英文逗号”分割
  ckeywords={量子信息, 量子计算, 金刚石色心, 微波谐振腔, 超导量子比特},
  ekeywords={Quantum information, quantum computation, NV center, microwave resonator, superconducting qubit}
}

% 定义中英文摘要和关键字
\begin{cabstract}
  本文对基于超导量子比特与固态自旋的量子系统进行了探究。本文首先对对于这方面已有的工作与研究进展进行了调研与总结,并简要给出了相关理论基础。对于自旋与谐振腔的耦合强度,本文进行了仿真与讨论,对于如平面波导谐振腔,改良的三维谐振腔等谐振腔类型得到了与现有文献中相符的结论。本文对新型2.5维谐振腔与单个自旋的耦合,参考现有方案进行了仿真,设计的改进,制备方案的整理与实施,测量系统的搭建、测试与使用。本文对上述一系列实验工作进行了整理与总结。由于时间有限,实验部分暂时还未得出理想的结果,因此对于可能的问题进行了讨论与排除,并计划了下一步的实验方案。
\end{cabstract}

% 如果习惯关键字跟在摘要文字后面,可以用直接命令来设置,如下:
% \ckeywords{\TeX, \LaTeX, CJK, 模板, 论文}

\begin{eabstract}
  This work explores hybrid quantum system based on Superconducting qubits and solid state spins. Existing research and progress in this field is firstly summarized and the basic theory is briefly presented. For the coupling strength between spins and microwave cavities, simulations are carried out with discussions. Results from the simulations about CPW resonators and modified 3D cavities agree with existing results in the literature. The coupling between modified 2D resonator and single spin is also estimated from simulation based on existing design. The design for the modified 2D resonator (2.5D resonator) is further optimized for the fabrication process. The required measurement system is developed, characterized and applied for the measurement of the 2.5D resonator. This work summarizes the above experimental progress. Expected measurement results of the 2.5D resonator are not yet completely obtained due to the limited time. Possible problems are discussed and excluded. The next step of the experiment is planned.
\end{eabstract}

% \ekeywords{\TeX, \LaTeX, CJK, template, thesis}
